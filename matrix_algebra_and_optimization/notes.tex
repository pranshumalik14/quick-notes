\documentclass[10pt]{article}
\usepackage{setup}

%%%%%%%%%%%%%%%%%%%%%%%%%%%%%%%%%%%%%
%       Article begins here         %
%%%%%%%%%%%%%%%%%%%%%%%%%%%%%%%%%%%%$

\date{Fall Semester, 2019}
\begin{document}

\title{\textbf{\Large{\textsc{ECE367:} Matrix Algebra and Optimization}} \\ \Large{Reference Notes}\vspace{-0.3cm}}
\author{Joonsu Oh\\\footnotesize Typeset by Pranshu Malik}

\maketitle
\tableofcontents
\blfootnote
{
    \textbf{Note:} This document is \underline{not} meant to be a comprehensive treatment of the material 
    or contain the complete course notes. The primary purpose of this document is to summarise key concepts along with some analysis and proofs, that can aid in revising the course or serve as a quick reference to a particular topic. Any suggestions or corrections are appreciated and can be
    sent directly to \texttt{\href{mailto:pranshu.malik@mail.utoronto.ca}{pranshu.malik@mail.utoronto.ca}}
}

\pagebreak

\section{Optimization Problems in Standard Form}

A typical optimization problem formulation is as follows:\\

\begin{defn}{}
    For decision variables $x_1$, $x_2$, \ldots, $x_n$
\end{defn}

\section{Mathematical Preliminaries}

\subsection{Sets}
\subsection{Functions}
\subsection{Fields}

\section{Vector Spaces}
\subsection{Subspaces}
\subsection{Affine Sets}

\section{Norms and Inner Products}
\subsection{Norms}
\subsection{Inner Products}

\section{Projection on Subspaces}
\subsection{Orthogonal Projection on One-dimensional Subspaces}
\subsection{Induced Norm}
\subsection{Metrics and Induced Metrics}
\subsection{Projection on Subspaces}
\subsection{Computing Projections with General Basis}
\subsection{Computing Projections with Orthonormal Basis}
\subsection{The Gram-Schmidt Procedure}
\subsection{Projection on the Orthogonal Complement}
\subsection{Projection on Affine Sets, including Lines and Hyperplanes}

\section{Linear Transformations}
\subsection{The Matrix of a Linear Transformation}
\subsection{The Fundamental Theorem of Linear Algebra}

\section{Operators}
\subsection{Invariant Subspaces}
\subsection{Upper Triangular Matrices for Operators on Complex Vector Spaces}
\subsubsection{Theorems for Operators}
\subsubsection{Change of Basis}
\subsubsection{Theorems for Matrices}
\subsection{Diagonal Matrices for Operators}

\section{Operators on Inner-Product Spaces}
\subsection{Upper Triangular Matrices}
\subsection{Linear Functionals and Adjoints}
\subsection{Operators on Inner Product Spaces}
\subsubsection{Self-Adjoint Operators}
\subsubsection{Normal Operators}
\subsection{The Spectral Theorem}
\subsubsection{Complex Inner-Product Spaces}
\subsubsection{Real Inner-Product Spaces}
\subsection{Positive Semidefinite Operators}
\subsection{Isometries}
\subsection{Polar Decomposition}

\section{The Singular Value Decomposition}
\subsection{Singular Values}
\subsection{Singular Value Decomposition for Operators}
\subsection{Singular Value Decomposition for Linear Transformations}
\subsection{Rayleigh Quotients}
\subsection{Projection Matrices}
\subsection{Least Squares and the Pseudoinverse}
\subsubsection{Problem Setup}
\subsubsection{Solution Set} 
\subsubsection{The Moore-Penrose Pseudoinverse}
\subsubsection{Back to Least Squares}
\subsubsection{Coordinate View}

\section{Acknowledgements}

\end{document}